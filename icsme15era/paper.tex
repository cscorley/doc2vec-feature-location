\documentclass[conference]{IEEEtran}

\makeatletter
% IEEEtran.cls defines \labelindent for backward compatibility reasons
% Undefine \labelindent to allow the use of package enumitem
\let\labelindent\relax
\makeatother


\usepackage[font=small]{caption}
\usepackage[T1]{fontenc}
\usepackage[para]{footmisc}
\usepackage[pdftex]{graphicx}
\usepackage[utf8]{inputenc}
\usepackage{array}
\usepackage{balance}
\usepackage{booktabs}
\usepackage{cite}
\usepackage{color}
\usepackage{comment}
\usepackage{enumitem}
\usepackage{framed}
\usepackage{listings}
\usepackage{microtype}
\usepackage{subcaption}
\usepackage{url}

% makes floatrow put crap on the same row


% SQUEEZE
%\addtolength{\parskip}{-1pt}


\definecolor{lightred}{RGB}{150,0,0}
\definecolor{lightgreen}{RGB}{0,150,0}
\definecolor{lightblue}{RGB}{0,0,150}

\lstdefinelanguage{diff}{
  morecomment=[f][\color{lightblue}]{diff },
  morecomment=[f][\color{lightblue}]{index },
  morecomment=[f][\color{lightblue}]{@@},     % group identifier
  morecomment=[f][\color{lightred}]-,         % deleted lines
  morecomment=[f][\color{lightgreen}]+,       % added lines
  morecomment=[f][\color{lightblue}]{---},    % Diff header lines (must appear after +,-)
  morecomment=[f][\color{lightblue}]{+++},
}
\hyphenation{}

\newcommand{\attn}[1]{{\color{red}#1}}
\newcommand{\desc}[1]{{\emph{\color{blue}#1}}}
\newcommand{\needcite}{\attn{\tiny{[cite]}}}
\newcommand{\todo}[1]{\strut\smash{\colorbox{yellow}{\bf TODO: #1}}}
\setlength\OuterFrameSep{0.5em}
\setlength\FrameSep{0.5em}

\clubpenalty = 10000
\widowpenalty = 10000
\displaywidowpenalty = 10000


\newcommand{\dv}{{DV}}

\begin{document}
\title{Exploring the Use of Deep Learning\\ for Feature Location}
\author{
    \IEEEauthorblockN{
        Christopher S.\ Corley
    }
    \IEEEauthorblockA{
        The University of Alabama\\
        Tuscaloosa, AL, USA\\
        cscorley@ua.edu
    }
    \and
    \IEEEauthorblockN{
        Kostadin Damevski
    }
    \IEEEauthorblockA{
        Virginia Commonwealth University\\
        Richmond, VA, USA\\
        damevski@acm.org
    }
    \and
    \IEEEauthorblockN{
        Nicholas A.\ Kraft
    }
    \IEEEauthorblockA{
        ABB Corporate Research\\
        Raleigh, NC, USA\\
        nicholas.a.kraft@us.abb.com
    }
}


\maketitle

\begin{abstract}

Deep learning models can infer complex patterns present in
natural language text. Relative to n-gram models, deep learning models can capture more
complex statistical patterns based on smaller training corpora. In
this paper we explore the use of a particular deep learning model,
document vectors (DVs), for feature location.  DVs seem well suited to
use with source code, because they both capture the influence of
context on each term in a corpus and map terms into a continuous
semantic space that encodes semantic relationships such as
synonymy. We present preliminary results that show that a feature
location technique (FLT) based on DVs can outperform an analogous FLT
based on latent Dirichlet allocation (LDA) and then suggest several
directions for future work on the use of deep learning models to
improve developer effectiveness in feature location.

\end{abstract}

\begin{IEEEkeywords}
deep learning;
neural networks;
document vectors;
feature location
\end{IEEEkeywords}

\section{Introduction}\label{introduction}
% vim:syntax=tex

Feature location is a frequent and fundamental activity for a developer tasked with changing a software system.
Whether a change task involves adding, modifying, or removing a feature, a developer cannot complete the task without first locating the source code that implements the feature.
The state-of-the-practice in feature location is to use an IDE tool based on keyword or regex search, but Ko et al.~\cite{Ko-etal:2006} observed such tools leading developers to failed searches nearly 90\% of the time.

The state-of-the-art in feature location~\cite{Dit-etal:2011} is to use a feature location technique (FLT) based, at least in part, on text retrieval (TR).
The standard methodology~\cite{Marcus-etal:2004} is to extract a document for each class or method in a source code snapshot, to train a TR model on those documents, and to create an index of the documents from the trained model.
Topics models (TMs)~\cite{Blei:2012} such as latent Dirichlet allocation (LDA)~\cite{Blei-etal:2003} are the state-of-the-art in TR and outperform vector-space models (VSMs) in the contexts of natural language~\cite{Deerwester-etal:1990,Blei-etal:2003} and source code~\cite{Poshyvanyk-etal:2007,Lukins-etal:2010}.
Yet, modern TMs such as online LDA~\cite{Hoffman-etal:2010} natively support only the online addition of a new document, whereas VSMs also natively support online modification or removal of an existing document.
So, TM-based FLTs provide the best accuracy, but unlike VSM-based FLTs, they require computationally-expensive retraining subsequent to source code changes.

Rao\cite{Rao:2013} proposed FLTs based on customizations of LDA and latent semantic indexing (LSI) that support online modification and removal.
%These FLTs require less-frequent retraining than others based on TMs.
These FLTs require less-frequent retraining than other TM-based FLTs,
but the remaining cost of periodic retraining inhibits their application to large software, and the reliance on customization hinders their extension to new TMs.

We envision an FLT that is: (1)~accurate like a TM-based FLT, (2)~inexpensive to update like a VSM-based FLT, and (3)~extensible to accommodate any off-the-shelf TR model that supports online addition of a new document.
Unfortunately, our vision is incompatible with the standard methodology for FLTs.
Existing VSM-based FLTs fail to satisfy the first criteria, and existing TM-based FLTs fail to satisfy the second or third criteria.
Indeed, given the current state-of-the-art in TR, it is impossible for an FLT to satisfy all three criteria while following the standard methodology.

In this paper we propose a new methodology for FLTs.
Our methodology is to extract a document for each changeset in the source code history and to train a TR model on the changeset documents, and then to extract a document for each class or method in a source code snapshot and to create an index of the class/method documents from the trained (changeset) model.
This new methodology stems from four key observations:
\begin{itemize}[leftmargin=*]
  \item
    Like a class/method definition, a changeset has program text.
  \item
    Unlike a class/method definition, a changeset is immutable.
  \item
    A changeset corresponds to a commit.
  \item
    An atomic commit involves a single feature.
\end{itemize}
It follows from the first two observations that it is possible for an FLT following our methodology to satisfy all three of the criteria above.
The next two observations influence the training and indexing steps of our methodology,
which have the conceptual effect of relating classes (or methods) to changeset topics.
By contrast, the training and indexing steps of the standard methodology
have the conceptual effect of relating classes to class topics (or methods to method topics).

To evaluate the new methodology, we used it to implement FLTs based on online LSI and online LDA.
We next used two benchmarks to compare the accuracy of these FLTs to the accuracy of comparable FLTs that follow the standard methodology.
Combined, the two benchmarks comprise over 1,200 defects and features from 14 open-source Java projects.
We then used a subset of over 600 defects and features to conduct a historical simulation that demonstrates how the FLTs perform as a project evolves.
Our evaluation results provide evidence that our new methodology is sound and indicate that FLTs following it provide similar accuracy to those following the standard methodology while eliminating retraining costs.

The remainder of the paper is organized as follows.
We first review background and related work (\S\ref{sec:related})
We next present our new methodology for FLTs (\S\ref{sec:changeset}) and report evaluation results for the online-LDA-based FLT (\S\ref{sec:study}).
We then conclude (\S\ref{sec:conclusion}).




\begin{comment}
Software developers are often confronted with maintenance tasks that involve
navigation of repositories that preserve vast amounts of project history.
Navigating these software repositories can be a time-consuming task, because
their organization can be difficult to understand.  A software developer who is
tasked with changing a large software system spends effort on program
comprehension activities to gain the knowledge needed to make the
change~\cite{Corbi:1989}.  Fortunately, topic models such as latent Dirichlet
allocation (LDA)~\cite{Blei-etal:2003} can help developers to navigate and
understand software repositories by discovering topics (word distributions) that
reveal the thematic structure of the
data~\cite{Linstead-etal:2007,Thomas-etal:2011,Hindle-etal:2014}.

One particular application of topic models is for \emph{feature location}.
Feature location is the act of identifying the source code that implements
a system feature.  The current state-of-the-practice for feature location is to
use a keyword search tool, such as \texttt{grep}.  Ko et al.~\cite{Ko-etal:2006}
show that developers fail using this type of searching upwards to 88\% of the
time.  Text retrieval techniques, such as topic modeling, show promise in
remedying this problem~\cite{Marcus-etal:2004}.

Typical topic-modeling-based feature location techniques (FLT) construct models
from corpora of text extracted from a source code
snapshot~\cite{Dit-etal:2011}.  To use a topic-modeling-based FLT, there are
generally two key steps: training and indexing.  In the first step, a corpus of
source code entities, such as methods or classes, are used to train the model to
learn word co-occurences within those entities.  The indexing step uses the
trained model to construct an index of the source code entities based on their
inferred topic distribution.  That is, an index is made of each source code's
\emph{thematic structure}, and not it's raw content.  Keeping such a model and
index up-to-date is expensive, because the frequency and scope of source code
changes, such as file removal, necessitate retraining the model on the updated
corpus and reindexing.  This situation is sub-optimal whether your perspective
is academic research or industrial tool-building.  Like Rao et
al.~\cite{Rao-etal:2013}, our primary research goal is elimination of this cost.
However, unlike Rao et al., we do not intend to develop new topic modeling
techniques, but rather use the existing ones.

In this paper, we propose a fresh take on topic-modeling-based FLTs by
leveraging online topic models and mining software repositories to construct
topic models that do not need retraining.  Online topic models do not need to
know the entire input corpus prior to
training~\cite{Hoffman-etal:2010}.  That is, online topic models can
be incrementally trained over time as more data becomes available.
Moreover, a version control repository, such as Git, keeps a history of source
code documents as they change over time.  These changes are represented as
changesets, which provide concise views of the differences between two revisions
of the same document.  By training an online topic model on changesets and
indexing the source code on that model, we can stream documents (i.e.,
changesets) from the version control repository to incrementally train the topic
model.  This enables searching over the current source code index without
retraining an entirely new model.

In our previous work~\cite{Corley-etal:2014}, we show that topic models trained
on changesets produce topics which have comparable topic distinctness
scores~\cite{Thomas-etal:2011} as topic models trained on snapshots.  Further,
we show that the corpora express the same frequency of words.  We expand the
work to demonstrate the effectiveness of changeset topic modeling for feature
location and report on an empirical study in which we investigate the
feasibility of this approach.
We define a LDA-based FLT using changesets.  We combine two benchmarks totaling
over 1200 defects and features from fourteen open source Java projects.  We also
present a \emph{historical simulation} that approximates how the FLT would perform
throughout the evolution of a project.

Our results show that the changeset approach is feasible and has performace
comparable to the snapshot approach.  In many cases the changeset approach
out-performs current snapshot approaches, but is no silver bullet.  We argue
that the evidence suggests that changeset-based topic modeling warrants further
investigation and adoption.  Additionally, the historical simulation suggests that
current evaluation approaches do not accurately capture the true FLT
performance.

This paper makes the following contributions:

\begin{itemize} \item An approach for using changesets for feature location
        \item A empirical study of fourteen open source Java projects \item
            Towards increasing open science principles in software engineering:
            the complete project history, source code, and an updated dataset
            for replication of this study.  \end{itemize}

The remainder of the paper is organized as follows.  We first review background
and related work (\S\ref{sec:related}) before introducing our new
changeset-based FLT (\S\ref{sec:changeset}).  We next discuss our case study
(\S\ref{sec:study}), which spans fourteen open source Java projects.  We then
conclude (\S\ref{sec:conclusion}).
\end{comment}



\section{Background}\label{background}

%   i.) Describe all the steps of a Deep Learning FL system
%
%

%how do FL systems work in general?
%
Feature location systems retrieve a ranked list of program elements (e.g.
methods, classes) for a developer query. In the {\em training} phase, feature
location systems commonly construct a model of the software, at the granularity
of program elements, based on the natural language embedded in identifiers and
comments. In the {\em retrieval} phase, given a natural language query, feature
location systems use the model to retrieve all of the relevant program elements
with high similarity to the query.

\subsection{Feature Location Workflow}

%how does a deep learning system differ
%
A feature location system based on deep learning, during its training phase,
creates a contextual representation of the natural language terms embedded in
the source code. This contextual representation includes influence from terms
preceding and following each term, relative to their distance from that term.
More intuitively, such models incorporate mutual influence between terms in the
same method, while terms that are closer in distance (e.g. occur the same
statement) influence each other more strongly.


\begin{figure}[tb]
\centering
\includegraphics[width=.9\columnwidth]{figures/neuralnet.pdf}
\caption{A deep learning neural network encodes source code identifiers, in the
order they appear in the source code, in its input layer. Using a deep
structure of hidden layers, each term and its context receives a semantic
vector representation. The output layer consists of vector for each term in
the corpus; the vector feature size is arbitrary and does not need to relate to
the number of terms in the corpus.}
\label{fig:neuralnet}
\vspace*{-2mm}
\end{figure}


%what is deep learning
Deep learning is based on a multi-stage neural network, consisting of several
hidden layers in addition to single input and output layers.  The input layer
consists of an ordered sequences of identifiers extracted from the code. The
multiple hidden layers serve to capture the context for each encountered term,
representing the complex patterns of term contexts occurring in the corpus. The
output layer consists of a vector for each term, which has been shown to carry
semantic meaning. An example of this architecture for a single line of code is
shown in Figure~\ref{fig:neuralnet}. Recent advances in this area have stemmed
from the use of novel neural network architectures, including recurrent neural
networks that connect the hidden layers back to the input layer, among other
strategies. The systems are trained using backpropagation and gradient descent,
techniques common to many neural network based models.

%paragraph2vec
An extension to learning the semantic vector representation of words is the use
of an additional vector that will encode the representation of a larger body of
text, such as a paragraph or an entire document~\cite{le_distributed_2014}.
While comparing word vectors indicates semantic relations between two terms,
comparing two document vectors carries a similar semantic connotation at the
document level. For instance, the approach has been applied for determining the
sentiment (i.e. positive, negative) of reviews on a popular movie recommendation
site.

%preprocessing
A number of preprocessing steps are commonly performed before the training phase
of feature location systems.  
The steps commonly used are~\cite{Marcus-etal_2004,Marcus-Menzies_2010}: % that we use are:
\begin{itemize}
    \item {\it Splitting}: separate tokens into constituent words based on
        common coding style conventions (e.g., the use of camel case or
        underscores) and on the presence of non-letters (e.g., punctuation or
        digits)
    \item {\it Normalizing}: replace each upper case letter with the
        corresponding lower case letter
    \item {\it Filtering}: remove common words such as articles (e.g., `an' or
        `the'), programming language keywords, standard library entity names, or
        short words
\end{itemize}


%retrieval
During the retrieval state of feature location, a similarity measure (e.g.
cosine similarity) between the words in the query and words in each program
element is computed. The program elements are ranked based on this similarity
metric and presented to the developer in descending order.

If document vectors are used then a special inference process is needed to infer
a vector representation for the entire query, treated as a document in the
corpus. Following this, the vectors of the query and each program element (i.e.
method or class) can be compared to produce the final ranking.

\subsection{Semantic Similarity}


\begin{table*}[tb]
\centering
\small
\caption{Examples of semantically similar terms and their weight for a deep model trained on
the ArgoUML code base. Only terms with weight > 0.6 are included.}
\label{tab:semsim}
\begin{tabular}{|p{0.30\textwidth}|p{0.60\textwidth}|}
\hline \hline
{\em Term(s)} & {\em Semantically Similar Terms and Weight}\\ \hline \hline
association & (roles 0.75), (role 0.72), (classifier 0.72), (connection, 0.61) \\ \hline
save & (saved 0.69), (pcs 0.64), (exists 0.63), (projects 0.60), (close 0.61), (file 0.60) \\ \hline
file & (filter 0.78), (zip 0.74), (exists 0.74), (persister 0.71), (files 0.69), (directory 0.69) \\ \hline
file + save & (exists 0.77), (saved 0.74), (filter 0.73), (zip 0.72), (unable 0.67), (projects 0.67), (persister 0.67), (files 0.66), (cant 0.66), (scheme 0.65) \\ \hline
explorer + diagrams - creating & (nodes 0.71), (deletion 0.67), (perspectives 0.65), (perspective 0.60), (updated 0.60), (modified 0.60)\\
\hline \hline
\end{tabular}
\end{table*}

The result of the deep learning models described in this paper are vectors
representing each term in the corpus~\cite{mikolov_distributed_2013}. Similar
vectors can also be computed to represent an entire paragraph or
document~\cite{le_distributed_2014}. Semantic similarity is the notion that
these vectors are composable semantically, e.g., the result of the operation
{\em vec(``The Eiffel Tower'') - vec(``Paris'') + vec(``London'')} is closest to
{\em vec(``Big Ben'')}. This capability is established automatically by the
system, without any additional processing or supervised input.

In feature location, as in general information retrieval, the retrieval quality
can only be as good as the quality of the developer query. Problems such as the
dictionary mismatch problem, as well as the propensity of users to issue short
queries, have previously been observed as common difficulties in using feature
location tools in the field~\cite{haiduc_effect_2011, damevski_field_2015}.
Semantic similarity can be a useful capability in mitigating these problems, by
performing query recommendation that allows the user to extend his or her
queries with terms from the same corpus, or automatic query extension.

To illustrate this capability for source code-based corpora, we provide a set of
illustrative examples gathered on the ArgoUML v0.22 in Table~\ref{tab:semsim}.
In many cases semantic similarity provides reasonable results for similar terms,
though exceptions exist largely due to the limited appearance of certain words
in the corpus.  We anticipate that larger corpora, based on larger code bases,
or leveraging related code bases or documents, could improve the results of
semantic similarity even further.




\section{Preliminary Study}\label{study}
% vim:syntax=tex

In this section we describe the design of a case study in which we
compare topic models trained on changesets to those trained on snapshots.
%explore the relationship between ownership and linguistic topics in source code.
We describe the case study using the Goal-Question-Metric approach~\cite{Basili-etal:94}.
% TODO
%The data for the case study is available in this paper's online
%appendix\footnote{\url{xxxxxx}}.

\subsection{Definition and Context}

% TODO
Our \textit{goal} is to ... 
The \textit{quality focus} of the study is on informing development
decisions and policy changes that could lead to software with fewer
defects.
The \textit{perspective} of the study is of a researcher, developer, or
project manager who wishes to gain understanding of the concepts or
features implemented in the source code.
The \textit{context} of the study spans the version histories of ...
open source systems.

Toward achievement of our goal, we pose the following research questions:
\begin{description}[font=\itshape\mdseries,leftmargin=10mm,style=sameline]
% TODO
    \item[RQ1] .... 
\end{description}
At a high level, we want to...
In the remainder of this section we introduce the subjects of our study,
describe the setting of our study, and report our data collection and analysis procedures.

%%%%%%%%%%%%%%%%%%%%%%%%%%%%%%%%%%%%%%%%%%%%%%%%%%%%%%%%%%%%%%%%%%%%%%%%

\subsection{Subject software systems}

% TODO

%%%%%%%%%%%%%%%%%%%%%%%%%%%%%%%%%%%%%%%%%%%%%%%%%%%%%%%%%%%%%%%%%%%%%%%%

\subsection{Setting}

\begin{figure*}[!th]
    \centering
    \includegraphics[width=.75\textwidth]{changeset}
    \caption{Extraction and Modeling Process}
    \label{fig:process}
\vspace{-10pt}
\end{figure*}

Our document extraction process is shown on the left side of Figure~\ref{fig:process}.
We implemented our document extractor in Python v2.7
using the Dulwich library\footnote{\url{http://www.samba.org/~jelmer/dulwich/}}. %\footnote{\url{https://pypi.python.org/pypi/dulwich}}
We extract documents from both a snapshot of the repository at a tagged
release and each commit reachable from that tag's commit.
The same preprocessing steps are employed on all documents extracted.

% TODO
For our document extraction from a snapshot, we ...

To extract text from the changesets, we look at the output of viewing
the \texttt{git diff} between two commits.
Figure~\ref{fig:diff} shows an example of what a changeset might look
like in Git.
In our changeset text extractor, we only extract all text related to the
changed file, e.g., context, removed, and added lines.  
Metadata lines are ignored.
Note that we do not consider where the text originates from,
only that it is text changed by the commit.

After extracting tokens, we split them based on camel case, underscores, and non-letters.
We normalize to lower case before filtering non-letters, English stop words~\cite{StopWords}, Java keywords, and words shorter than three characters long.
We do not stem words.

Our modeling generation is shown on the right side of Figure~\ref{fig:process}.
We implemented our modeling using the Python library Gensim~\cite{Gensim}.
Gensim's LDA implementation is based on an Online LDA by Hoffman et al.~\cite{Hoffman-etal:2010}
and uses variational inference instead of a Collapsed Gibbs Sampler.
Unlike Gibbs sampling, in order to ensure that the model converges for each document,
we allow LDA to see each document $10$ times by setting Gensim's initialization parameter \texttt{passes} to this value.
% TODO
We set the following LDA parameters for all ... systems:
$100$ topics ($K$),
a symmetric $\alpha=0.01$,
$\beta$ is left as a default value of $1/K$ (also $0.01$).


\begin{figure}[ht]
\centering
\footnotesize
\begin{lstlisting}[language=diff, basicstyle=\ttfamily]
diff --git a/lao b/tzu
index 635ef2c..5af88a8 100644
--- a/lao
+++ b/tzu
@@ -1,7 +1,6 @@
-The Way that can be told of is not the eternal Way;
-The name that can be named is not the eternal name.
 The Nameless is the origin of Heaven and Earth;
-The Named is the mother of all things.
+The named is the mother of all things.
+
 Therefore let there always be non-being,
   so we may see their subtlety,
 And let there always be being,
@@ -9,3 +8,6 @@ And let there always be being,
 The two are the same,
 But after they are produced,
   they have different names.
+They both may be called deep and profound.
+Deeper and more profound,
+The door of all subtleties!
\end{lstlisting}
\caption{Example of a \texttt{git diff}. Black or blue lines denote metadata about the change useful for patching, red lines (beginning with a single~\texttt{-}) denote line removals, and green lines (beginning with a single~\texttt{+}) denote line additions.}
\label{fig:diff}
\vspace{-10pt}
\end{figure}


%%%%%%%%%%%%%%%%%%%%%%%%%%%%%%%%%%%%%%%%%%%%%%%%%%%%%%%%%%%%%%%%%%%%%%%%

\subsection{Data Collection and Analysis}

% TODO
We create two corpora for each of our four subject systems.
We then used LDA to model the documents into topics.

To answer RQ1, ... 


%%%%%%%%%%%%%%%%%%%%%%%%%%%%%%%%%%%%%%%%%%%%%%%%%%%%%%%%%%%%%%%%%%%%%%%%

\subsection{Results}

% TODO
RQ1 asks ...


\section{Related Work}\label{related}
% vim:syntax=tex

In this section we provide an overview of two topic models,
latent semantic indexing (LSI) and latent Dirichlet allocation (LDA),
and review closely related work.

\subsection{Latent Semantic Indexing}

Latent semantic indexing~\cite{Deerwester:1990} is an indexing and 
retrieval methodolgy. LSI uses a statistical technique, singular value 
decomposition to identify patterns within the unstructured data. That is, 
LSI identifies relationships between terms and documents, and places 
documents that are related close to one another creating a semantic space. 


\subsection{Latent Dirichlet Allocation}

Latent Dirichlet allocation~\cite{Blei-etal:2003} is a generative topic model.
LDA models each document in a corpus of discrete data as a finite mixture over a set of topics
and models each topic as an infinite mixture over a set of topic probabilities.
That is, LDA models each document as a probability distribution
indicating the likelihood that it expresses each topic and
models each topic that it infers as a probability distribution
indicating the likelihood of a word from the corpus being assigned to the topic.

Inputs to LDA include a corpus and $K$, the number of topics.
LDA represents each document in the corpus as a bag-of-word (multiset)
and thus disregards word order and structure.
Outputs of LDA include $\phi$, the term-topic probability distribution,
and $\theta$, the topic-document probability distribution.


\subsection{Feature Location}



\section{Conclusions and Future Work}\label{conlusion}
% vim:syntax=tex

In this paper we conducted a study on modeling the topics of changesets in comparison to the traditional snapshot approach.
We use latent Dirichlet allocation (LDA) to extract linguistic topics from
changesets and snapshots (releases).

We addressed two research questions regarding the topic modeling of changesets.
First, we compare a batch topic-modeling-based FLT trained on the changesets
of a project's history to one trained on the snapshot of source code entities.
Second, we compare a batch topic-modeling-based FLT trained on changesets
to a temporal topic-modeling-based FLT trained on the same changesets over time.
We found that changesets can perform as well as or better than snapshots.
We also show that temporal analysis more accurately portrays how a FLT would execute in a real environment.


Future work includes deploying this appoach in a development environment.
Since the source to our approach is online, we encourage other researchers
to investigate this future work as well.
We also would like to expand the temporal parts of this study to include
both snapshots and changesets.
It would be particularly useful to compare results between batch snapshots and temporal snapshots.
Additional future work includes expanding our study to other systems,
particularly ones that are not Java.
It seems unlikely that our results are specific to Java systems,
though we cannot confirm this assumption without experimentation.




%\section*{Acknowledgments}
%We acknowledge noone.

\bibliographystyle{IEEEtran-nourl}
\bibliography{doc2vec}

\end{document}
