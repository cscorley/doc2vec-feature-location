\documentclass[conference]{IEEEtran}

\makeatletter
% IEEEtran.cls defines \labelindent for backward compatibility reasons
% Undefine \labelindent to allow the use of package enumitem
\let\labelindent\relax
\makeatother

\usepackage[utf8]{inputenc}
\usepackage[T1]{fontenc}
\usepackage{microtype}

\usepackage{array}
\usepackage{balance}
\usepackage{booktabs}
\usepackage{cite}
\usepackage{color}
\usepackage{comment}
\usepackage{enumitem}
\usepackage[para]{footmisc}
\usepackage{framed}
\usepackage[pdftex]{graphicx}
\usepackage{listings}
\usepackage{subcaption}
\usepackage{url}


% SQUEEZE
%\addtolength{\parskip}{-1pt}


\definecolor{lightred}{RGB}{150,0,0}
\definecolor{lightgreen}{RGB}{0,150,0}
\definecolor{lightblue}{RGB}{0,0,150}

\lstdefinelanguage{diff}{
  morecomment=[f][\color{lightblue}]{diff },
  morecomment=[f][\color{lightblue}]{index },
  morecomment=[f][\color{lightblue}]{@@},     % group identifier
  morecomment=[f][\color{lightred}]-,         % deleted lines
  morecomment=[f][\color{lightgreen}]+,       % added lines
  morecomment=[f][\color{lightblue}]{---},    % Diff header lines (must appear after +,-)
  morecomment=[f][\color{lightblue}]{+++},
}
\hyphenation{}

\newcommand{\attn}[1]{{\color{red}#1}}
\newcommand{\desc}[1]{{\emph{\color{blue}#1}}}
\newcommand{\needcite}{\attn{\tiny{[cite]}}}
\newcommand{\todo}[1]{\strut\smash{\colorbox{yellow}{\bf TODO: #1}}}

\begin{document}
\title{Modeling Changeset Topics for Feature Location}
\author{
    \IEEEauthorblockN{
        Christopher S.\ Corley,
        Kelly L.\ Kashuda
    }
    \IEEEauthorblockA{
        The University of Alabama\\
        Tuscaloosa, AL, USA\\
        \{cscorley, klkashuda\}@ua.edu
    }

    \and

    \IEEEauthorblockN{
        Nicholas A.\ Kraft
    }
    \IEEEauthorblockA{
        ABB Corporate Research\\
        Raleigh, NC, USA\\
        nicholas.a.kraft@us.abb.com
    }
}


\maketitle

\begin{abstract}
Feature location is a program comprehension activity in which a developer
inspects source code to locate the classes or methods that implement a feature of interest.
Many feature location techniques (FLTs) are based on text retrieval models, and
in such FLTs it is typical for the models to be trained on source code snapshots.
However, source code evolution leads to model obsolescence and
thus to the need to retrain the model from the latest snapshot.
In this paper, we introduce a topic-modeling-based FLT in which the model
is built incrementally from source code history.
By training an online learning algorithm using changesets, the FLT
maintains an up-to-date model without incurring the non-trivial computational cost associated with retraining traditional FLTs.
Overall, we studied over 1,200 defects and features from 14 open-source Java projects.
We also present a historical simulation that demonstrates how the FLT performs as a project evolves.
Our results indicate that the accuracy of a changeset-based FLT is similar to that of a snapshot-based FLT, but without the retraining costs.
\end{abstract}

\begin{IEEEkeywords}
program comprehension;
feature location;
topic modeling;
mining software repositories;
changesets
\end{IEEEkeywords}

\section{Introduction}
\label{sec:intro}
% vim:syntax=tex

Feature location is a frequent and fundamental activity for a developer tasked with changing a software system.
Whether a change task involves adding, modifying, or removing a feature, a developer cannot complete the task without first locating the source code that implements the feature.
The state-of-the-practice in feature location is to use an IDE tool based on keyword or regex search, but Ko et al.~\cite{Ko-etal:2006} observed such tools leading developers to failed searches nearly 90\% of the time.

The state-of-the-art in feature location~\cite{Dit-etal:2011} is to use a feature location technique (FLT) based, at least in part, on text retrieval (TR).
The standard methodology~\cite{Marcus-etal:2004} is to extract a document for each class or method in a source code snapshot, to train a TR model on those documents, and to create an index of the documents from the trained model.
Topics models (TMs)~\cite{Blei:2012} such as latent Dirichlet allocation (LDA)~\cite{Blei-etal:2003} are the state-of-the-art in TR and outperform vector-space models (VSMs) in the contexts of natural language~\cite{Deerwester-etal:1990,Blei-etal:2003} and source code~\cite{Poshyvanyk-etal:2007,Lukins-etal:2010}.
Yet, modern TMs such as online LDA~\cite{Hoffman-etal:2010} natively support only the online addition of a new document, whereas VSMs also natively support online modification or removal of an existing document.
So, TM-based FLTs provide the best accuracy, but unlike VSM-based FLTs, they require computationally-expensive retraining subsequent to source code changes.

Rao\cite{Rao:2013} proposed FLTs based on customizations of LDA and latent semantic indexing (LSI) that support online modification and removal.
%These FLTs require less-frequent retraining than others based on TMs.
These FLTs require less-frequent retraining than other TM-based FLTs,
but the remaining cost of periodic retraining inhibits their application to large software, and the reliance on customization hinders their extension to new TMs.

We envision an FLT that is: (1)~accurate like a TM-based FLT, (2)~inexpensive to update like a VSM-based FLT, and (3)~extensible to accommodate any off-the-shelf TR model that supports online addition of a new document.
Unfortunately, our vision is incompatible with the standard methodology for FLTs.
Existing VSM-based FLTs fail to satisfy the first criteria, and existing TM-based FLTs fail to satisfy the second or third criteria.
Indeed, given the current state-of-the-art in TR, it is impossible for an FLT to satisfy all three criteria while following the standard methodology.

In this paper we propose a new methodology for FLTs.
Our methodology is to extract a document for each changeset in the source code history and to train a TR model on the changeset documents, and then to extract a document for each class or method in a source code snapshot and to create an index of the class/method documents from the trained (changeset) model.
This new methodology stems from four key observations:
\begin{itemize}[leftmargin=*]
  \item
    Like a class/method definition, a changeset has program text.
  \item
    Unlike a class/method definition, a changeset is immutable.
  \item
    A changeset corresponds to a commit.
  \item
    An atomic commit involves a single feature.
\end{itemize}
It follows from the first two observations that it is possible for an FLT following our methodology to satisfy all three of the criteria above.
The next two observations influence the training and indexing steps of our methodology,
which have the conceptual effect of relating classes (or methods) to changeset topics.
By contrast, the training and indexing steps of the standard methodology
have the conceptual effect of relating classes to class topics (or methods to method topics).

To evaluate the new methodology, we used it to implement FLTs based on online LSI and online LDA.
We next used two benchmarks to compare the accuracy of these FLTs to the accuracy of comparable FLTs that follow the standard methodology.
Combined, the two benchmarks comprise over 1,200 defects and features from 14 open-source Java projects.
We then used a subset of over 600 defects and features to conduct a historical simulation that demonstrates how the FLTs perform as a project evolves.
Our evaluation results provide evidence that our new methodology is sound and indicate that FLTs following it provide similar accuracy to those following the standard methodology while eliminating retraining costs.

The remainder of the paper is organized as follows.
We first review background and related work (\S\ref{sec:related})
We next present our new methodology for FLTs (\S\ref{sec:changeset}) and report evaluation results for the online-LDA-based FLT (\S\ref{sec:study}).
We then conclude (\S\ref{sec:conclusion}).




\begin{comment}
Software developers are often confronted with maintenance tasks that involve
navigation of repositories that preserve vast amounts of project history.
Navigating these software repositories can be a time-consuming task, because
their organization can be difficult to understand.  A software developer who is
tasked with changing a large software system spends effort on program
comprehension activities to gain the knowledge needed to make the
change~\cite{Corbi:1989}.  Fortunately, topic models such as latent Dirichlet
allocation (LDA)~\cite{Blei-etal:2003} can help developers to navigate and
understand software repositories by discovering topics (word distributions) that
reveal the thematic structure of the
data~\cite{Linstead-etal:2007,Thomas-etal:2011,Hindle-etal:2014}.

One particular application of topic models is for \emph{feature location}.
Feature location is the act of identifying the source code that implements
a system feature.  The current state-of-the-practice for feature location is to
use a keyword search tool, such as \texttt{grep}.  Ko et al.~\cite{Ko-etal:2006}
show that developers fail using this type of searching upwards to 88\% of the
time.  Text retrieval techniques, such as topic modeling, show promise in
remedying this problem~\cite{Marcus-etal:2004}.

Typical topic-modeling-based feature location techniques (FLT) construct models
from corpora of text extracted from a source code
snapshot~\cite{Dit-etal:2011}.  To use a topic-modeling-based FLT, there are
generally two key steps: training and indexing.  In the first step, a corpus of
source code entities, such as methods or classes, are used to train the model to
learn word co-occurences within those entities.  The indexing step uses the
trained model to construct an index of the source code entities based on their
inferred topic distribution.  That is, an index is made of each source code's
\emph{thematic structure}, and not it's raw content.  Keeping such a model and
index up-to-date is expensive, because the frequency and scope of source code
changes, such as file removal, necessitate retraining the model on the updated
corpus and reindexing.  This situation is sub-optimal whether your perspective
is academic research or industrial tool-building.  Like Rao et
al.~\cite{Rao-etal:2013}, our primary research goal is elimination of this cost.
However, unlike Rao et al., we do not intend to develop new topic modeling
techniques, but rather use the existing ones.

In this paper, we propose a fresh take on topic-modeling-based FLTs by
leveraging online topic models and mining software repositories to construct
topic models that do not need retraining.  Online topic models do not need to
know the entire input corpus prior to
training~\cite{Hoffman-etal:2010}.  That is, online topic models can
be incrementally trained over time as more data becomes available.
Moreover, a version control repository, such as Git, keeps a history of source
code documents as they change over time.  These changes are represented as
changesets, which provide concise views of the differences between two revisions
of the same document.  By training an online topic model on changesets and
indexing the source code on that model, we can stream documents (i.e.,
changesets) from the version control repository to incrementally train the topic
model.  This enables searching over the current source code index without
retraining an entirely new model.

In our previous work~\cite{Corley-etal:2014}, we show that topic models trained
on changesets produce topics which have comparable topic distinctness
scores~\cite{Thomas-etal:2011} as topic models trained on snapshots.  Further,
we show that the corpora express the same frequency of words.  We expand the
work to demonstrate the effectiveness of changeset topic modeling for feature
location and report on an empirical study in which we investigate the
feasibility of this approach.
We define a LDA-based FLT using changesets.  We combine two benchmarks totaling
over 1200 defects and features from fourteen open source Java projects.  We also
present a \emph{historical simulation} that approximates how the FLT would perform
throughout the evolution of a project.

Our results show that the changeset approach is feasible and has performace
comparable to the snapshot approach.  In many cases the changeset approach
out-performs current snapshot approaches, but is no silver bullet.  We argue
that the evidence suggests that changeset-based topic modeling warrants further
investigation and adoption.  Additionally, the historical simulation suggests that
current evaluation approaches do not accurately capture the true FLT
performance.

This paper makes the following contributions:

\begin{itemize} \item An approach for using changesets for feature location
        \item A empirical study of fourteen open source Java projects \item
            Towards increasing open science principles in software engineering:
            the complete project history, source code, and an updated dataset
            for replication of this study.  \end{itemize}

The remainder of the paper is organized as follows.  We first review background
and related work (\S\ref{sec:related}) before introducing our new
changeset-based FLT (\S\ref{sec:changeset}).  We next discuss our case study
(\S\ref{sec:study}), which spans fourteen open source Java projects.  We then
conclude (\S\ref{sec:conclusion}).
\end{comment}



\section{Background \& Related Work}
\label{sec:related}
% vim:syntax=tex

In this section we provide an overview of two topic models,
latent semantic indexing (LSI) and latent Dirichlet allocation (LDA),
and review closely related work.

\subsection{Latent Semantic Indexing}

Latent semantic indexing~\cite{Deerwester:1990} is an indexing and 
retrieval methodolgy. LSI uses a statistical technique, singular value 
decomposition to identify patterns within the unstructured data. That is, 
LSI identifies relationships between terms and documents, and places 
documents that are related close to one another creating a semantic space. 


\subsection{Latent Dirichlet Allocation}

Latent Dirichlet allocation~\cite{Blei-etal:2003} is a generative topic model.
LDA models each document in a corpus of discrete data as a finite mixture over a set of topics
and models each topic as an infinite mixture over a set of topic probabilities.
That is, LDA models each document as a probability distribution
indicating the likelihood that it expresses each topic and
models each topic that it infers as a probability distribution
indicating the likelihood of a word from the corpus being assigned to the topic.

Inputs to LDA include a corpus and $K$, the number of topics.
LDA represents each document in the corpus as a bag-of-word (multiset)
and thus disregards word order and structure.
Outputs of LDA include $\phi$, the term-topic probability distribution,
and $\theta$, the topic-document probability distribution.


\subsection{Feature Location}



\section{Changeset Topic Modeling}
\label{sec:changeset}
% vim:syntax=tex

In this section we describe how a topic model-based feature location
technique can use changesets.

%diff --git a/lao b/tzu
%index 635ef2c..5af88a8 100644
%--- a/lao
%+++ b/tzu
\begin{figure}[t]
\centering
\footnotesize
\begin{lstlisting}[language=diff, basicstyle=\ttfamily]
--- lao
+++ tzu
@@ -1,7 +1,6 @@
-The Way that can be told of is not the eternal Way;
-The name that can be named is not the eternal name.
 The Nameless is the origin of Heaven and Earth;
-The Named is the mother of all things.
+The named is the mother of all things.
+
 Therefore let there always be non-being,
   so we may see their subtlety,
 And let there always be being,
@@ -9,3 +8,6 @@ And let there always be being,
 The two are the same,
 But after they are produced,
   they have different names.
+They both may be called deep and profound.
+Deeper and more profound,
+The door of all subtleties!
\end{lstlisting}
\caption{Example of a \texttt{git diff}.
Black or blue lines denote metadata about the change useful for patching.
In particular, black lines represent context lines (beginning with a single space).
Red lines (beginning with a single~\texttt{-}) denote line removals,
and green lines (beginning with a single~\texttt{+}) denote line additions.}
\label{fig:diff}
\vspace{-10pt}
\end{figure}


We use the following terminology to describe document extraction of changesets.
A \textit{diff} is a set of text which represents the differences between two texts.
A \textit{patch} is a set of instructions (i.e., diffs) that is used to transform one set of texts into another.
\textit{Context lines} denote text useful for transforming the text, but do not represent the differences.
\textit{Added lines} are lines which were added in order to transform the first text into the second.
Similarly, \textit{removed lines} are lines which are removed for this same purpose.
Figure~\ref{fig:diff} shows an example of what a changeset might look like.
A \textit{changeset}, ideally, represents a single feature modification,
addition, or deletion, which may crosscut many source code entities.
A \textit{commit} is a representation of a changeset in a version control system.
The terms changeset and commit are often used interchangeably.

The document extraction process for changesets remains mostly the same as covered in Section~\ref{sec:related}.
However, instead of extracting documents by parsing source code for identifiers, comments, and literals,
the changeset itself is parsed.
In a changeset it may be desirable to parse further for source code entities using island grammar parsing\needcite.
The same preprocessor transformations may also occur in changesets.

To leverage the online functionality of the topic models,
we intermix the model training and retrieval steps.
First, we initialize a model in online mode.
Then, as changes are made, the model is updated with the new changesets as they are committed.
That is, with changesets, we incrementally update a model and can query it at any moment.

We can use a dynamic programming to keep a separate $\theta_{snapshot}$
up-to-date as new changesets are added to the model.
That is, upon a update to the model, new inferences of on only the source code documents affected by this changeset are made.
Additionally, we can then query the model as needed 
and rank the results of that query against $\theta_{snapshot}$.
Note that we never care about infering a $\theta_{changeset}$ 
for the changeset documents on which the model is built.



\section{Study}
\label{sec:study}
% vim:syntax=tex

In this section we describe the design of a case study in which we
compare topic models trained on changesets to those trained on snapshots.
%explore the relationship between ownership and linguistic topics in source code.
We describe the case study using the Goal-Question-Metric approach~\cite{Basili-etal:94}.
% TODO
%The data for the case study is available in this paper's online
%appendix\footnote{\url{xxxxxx}}.

\subsection{Definition and Context}

% TODO
Our \textit{goal} is to ... 
The \textit{quality focus} of the study is on informing development
decisions and policy changes that could lead to software with fewer
defects.
The \textit{perspective} of the study is of a researcher, developer, or
project manager who wishes to gain understanding of the concepts or
features implemented in the source code.
The \textit{context} of the study spans the version histories of ...
open source systems.

Toward achievement of our goal, we pose the following research questions:
\begin{description}[font=\itshape\mdseries,leftmargin=10mm,style=sameline]
% TODO
    \item[RQ1] .... 
\end{description}
At a high level, we want to...
In the remainder of this section we introduce the subjects of our study,
describe the setting of our study, and report our data collection and analysis procedures.

%%%%%%%%%%%%%%%%%%%%%%%%%%%%%%%%%%%%%%%%%%%%%%%%%%%%%%%%%%%%%%%%%%%%%%%%

\subsection{Subject software systems}

% TODO

%%%%%%%%%%%%%%%%%%%%%%%%%%%%%%%%%%%%%%%%%%%%%%%%%%%%%%%%%%%%%%%%%%%%%%%%

\subsection{Setting}

\begin{figure*}[!th]
    \centering
    \includegraphics[width=.75\textwidth]{changeset}
    \caption{Extraction and Modeling Process}
    \label{fig:process}
\vspace{-10pt}
\end{figure*}

Our document extraction process is shown on the left side of Figure~\ref{fig:process}.
We implemented our document extractor in Python v2.7
using the Dulwich library\footnote{\url{http://www.samba.org/~jelmer/dulwich/}}. %\footnote{\url{https://pypi.python.org/pypi/dulwich}}
We extract documents from both a snapshot of the repository at a tagged
release and each commit reachable from that tag's commit.
The same preprocessing steps are employed on all documents extracted.

% TODO
For our document extraction from a snapshot, we ...

To extract text from the changesets, we look at the output of viewing
the \texttt{git diff} between two commits.
Figure~\ref{fig:diff} shows an example of what a changeset might look
like in Git.
In our changeset text extractor, we only extract all text related to the
changed file, e.g., context, removed, and added lines.  
Metadata lines are ignored.
Note that we do not consider where the text originates from,
only that it is text changed by the commit.

After extracting tokens, we split them based on camel case, underscores, and non-letters.
We normalize to lower case before filtering non-letters, English stop words~\cite{StopWords}, Java keywords, and words shorter than three characters long.
We do not stem words.

Our modeling generation is shown on the right side of Figure~\ref{fig:process}.
We implemented our modeling using the Python library Gensim~\cite{Gensim}.
Gensim's LDA implementation is based on an Online LDA by Hoffman et al.~\cite{Hoffman-etal:2010}
and uses variational inference instead of a Collapsed Gibbs Sampler.
Unlike Gibbs sampling, in order to ensure that the model converges for each document,
we allow LDA to see each document $10$ times by setting Gensim's initialization parameter \texttt{passes} to this value.
% TODO
We set the following LDA parameters for all ... systems:
$100$ topics ($K$),
a symmetric $\alpha=0.01$,
$\beta$ is left as a default value of $1/K$ (also $0.01$).


\begin{figure}[ht]
\centering
\footnotesize
\begin{lstlisting}[language=diff, basicstyle=\ttfamily]
diff --git a/lao b/tzu
index 635ef2c..5af88a8 100644
--- a/lao
+++ b/tzu
@@ -1,7 +1,6 @@
-The Way that can be told of is not the eternal Way;
-The name that can be named is not the eternal name.
 The Nameless is the origin of Heaven and Earth;
-The Named is the mother of all things.
+The named is the mother of all things.
+
 Therefore let there always be non-being,
   so we may see their subtlety,
 And let there always be being,
@@ -9,3 +8,6 @@ And let there always be being,
 The two are the same,
 But after they are produced,
   they have different names.
+They both may be called deep and profound.
+Deeper and more profound,
+The door of all subtleties!
\end{lstlisting}
\caption{Example of a \texttt{git diff}. Black or blue lines denote metadata about the change useful for patching, red lines (beginning with a single~\texttt{-}) denote line removals, and green lines (beginning with a single~\texttt{+}) denote line additions.}
\label{fig:diff}
\vspace{-10pt}
\end{figure}


%%%%%%%%%%%%%%%%%%%%%%%%%%%%%%%%%%%%%%%%%%%%%%%%%%%%%%%%%%%%%%%%%%%%%%%%

\subsection{Data Collection and Analysis}

% TODO
We create two corpora for each of our four subject systems.
We then used LDA to model the documents into topics.

To answer RQ1, ... 


%%%%%%%%%%%%%%%%%%%%%%%%%%%%%%%%%%%%%%%%%%%%%%%%%%%%%%%%%%%%%%%%%%%%%%%%

\subsection{Results}

% TODO
RQ1 asks ...


\section{Threats To Validity}
\label{sec:threats}
% vim:syntax=tex

Our study has limitations that impact the validity of our findings,
as well as our ability to generalize them.
We describe some of these limitations and their impacts.

Threats to construct validity concern the adequacy of the study procedure with regard to
measurement of the concepts of interest and can arise due to poor measurement design.
Threats to construct validity include the use of cosine similarity as our measure of similarity for corpora
and the use of topic distinctness to evaluate the topic models.

Threats to internal validity include possible defects in our tool chain and possible errors
in our execution of the study procedure,
the presence of which might affect the accuracy of our results and the conclusions we draw from them.
We controlled for these threats by testing our tool chain and by assessing the quality of our data.
Because we applied the same tool chain to all subject systems, any errors are systematic and are unlikely
to affect our results substantially.

Additionaly, we found errors within the datasets themselves that would be a threat to internal validity.
In particular, the Moreno et al. dataset included classes that had package names that were not valid.
For example, the Bookkeeper goldset for issue report 29 listed
\texttt{bookkeeper-server.src.main.java.org.apache.bookkeeper.bookie.EntryLogger}
for the class fixed by this change, while the actual fully-qualified name of this class is
\texttt{org.apache.bookkeeper.bookie.EntryLogger}.
We make the assumption that the authors of this dataset used the directory structure of the project to build the package names.
Manual correction was required as our tool parses the files and uses the package name given in the source file, not the directory structure.

Another threat to internal validity pertains to the value of parameters such as $K$ that we selected for all models trained.
We decided that the changeset and snapshot models should have the same parameters to help facilitate evaluation and comparison.
We argue that our study is not about selecting the best parameters,
but to show that our changeset topic-model-based FLT approach is reasonable.

Threats to external validity concern the extent to which we can generalize our results.
The subjects of our study comprise fourteen open source projects in Java,
so we cannot generalize our results to systems implemented in other languages.
However, the systems are of different sizes, are from different domains, and
have characteristics in common with those of systems developed in industry.



\section{Conclusion}
\label{sec:conclusion}
% vim:syntax=tex

In this paper we conducted a study on modeling the topics of changesets in comparison to the traditional snapshot approach.
We use latent Dirichlet allocation (LDA) to extract linguistic topics from
changesets and snapshots (releases).

We addressed two research questions regarding the topic modeling of changesets.
First, we compare a batch topic-modeling-based FLT trained on the changesets
of a project's history to one trained on the snapshot of source code entities.
Second, we compare a batch topic-modeling-based FLT trained on changesets
to a temporal topic-modeling-based FLT trained on the same changesets over time.
We found that changesets can perform as well as or better than snapshots.
We also show that temporal analysis more accurately portrays how a FLT would execute in a real environment.


Future work includes deploying this appoach in a development environment.
Since the source to our approach is online, we encourage other researchers
to investigate this future work as well.
We also would like to expand the temporal parts of this study to include
both snapshots and changesets.
It would be particularly useful to compare results between batch snapshots and temporal snapshots.
Additional future work includes expanding our study to other systems,
particularly ones that are not Java.
It seems unlikely that our results are specific to Java systems,
though we cannot confirm this assumption without experimentation.




% \section*{Acknowledgment}
% %We thank the anonymous reviewers for their insightful comments and helpful suggestions.
% This material is based upon work supported
% by the National Science Foundation under Grant No.\ 1156563. % REU grant for KLK

%\newpage \ \newpage
\bibliographystyle{IEEEtran}
\bibliography{paper}

\end{document}
